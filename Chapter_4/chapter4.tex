\chapter{Literature Study}
\emph{
Brief summary of this chapter.
}
\section{IoT}

\subsection{IoT Structure}

\subsection{IoT Readiness}

\subsection{Information Gaps}

\section{BI}

\subsection{Intelligence}

\section{Analytics/Data Mining/AI}
Artificial Intelligence (AI) is a field of study that focuses on creating machines that can act or react intelligently. The field of AI is very wide spread and contains a very large application domain.  One of the most popular fields within AI is Machine Learning (ML). \cite{Joshi2017ArtificialPython}


\subsection{Machine Learning}
ML focuses on specialized intelligence in contrast to general intelligence.  A ML solution is designed with a specific problem in mind and uses data to train a model to solve the problem. This is done by making predictions or detecting patterns. 
ML is divided into two main areas.  These are supervised learning and unsupervised learning.  

\subsubsection{Supervised Learning}
Supervised learning is the process of training a machine by using labeled data. Thus the training set contains features and labels corresponding to the desired output of the model.  The model can then predict labels using similar features with unknown labels.  \cite{Ben-David2014UnderstandingAlgorithms}


\subsubsection{Unsupervised Learning}

\subsection{Anomaly Detection}


\subsection{Time Series Analysis}

\section{Systems Engineering}

\section{Summary}